\documentclass{article}
\usepackage{ee105}

%% Makes figure labels bold
\usepackage[labelfont=bf]{caption}

\begin{document}

%% Prevent headings on the first page, since we're not using \maketitle
\thispagestyle{plain}

%% Title goes here!
\prelab{6}{Biasing Circuitry}
\name

\begin{enumerate}
		\begin{figure}[!htb]
			\input lab6_vsource2
			\centerline{\box\graph}
			\caption{Resistive divider ``voltage source''}
			\label{resvsource}
		\end{figure}
	\item Consider the resistor network shown in Figure \ref{resvsource}. Let $V_{CC} = \unit{10}{\volt}$, $R_1 = \unit{9.35}{\kilo\ohm}$, and $R_2 = \unit{650}{\ohm}$. We can turn this resistive divider into a voltage source with an output resistance by taking its Th\'{e}venin equivalent. Find the open circuit output voltage and the output resistance of this voltage source.
	\\~\\~\\~\\~\\~\\~\\~\\
	
		\begin{figure}[!htb]
			\input lab6_prelab_vsource1
			\centerline{\box\graph}
			\caption{BJT voltage source}
			\label{bjtvsource}
		\end{figure}
	\item Now, consider a BJT voltage source such as the one shown in Figure \ref{bjtvsource}. Size resistor $R_C$ to achieve an output voltage of \unit{650}{\milli\volt}. Let $V_{CC} = \unit{10}{\volt}$, $I_{S} = \unit{26.03}{\femto\ampere}$, and $V_{T} = \unit{26}{\milli\volt}$. Ignore the Early effect for this calculation.
	\\~\\~\\~\\~\\~\\~\\~\\
	\item Find the output impedance and the power dissipated by this BJT voltage source. \hint{Remember the definition of power, $P=IV$.}
	\\~\\~\\~\\~\\~\\~\\~\\
	\item If we were to resize the resistors of our resistive divider (Figure \ref{resvsource}) to achieve the same output impedance as the BJT voltage source given the same output voltage, what would be the values of the two resistors? How much power would it draw?
	\\~\\~\\~\\~\\~\\~\\~\\
	
	\begin{figure}[!htb]
		\input lab6_prelab_csource3
		\centerline{\box\graph}
		\caption{Resistor ``current source''}
		\label{lab6_csource3}
	\end{figure}
	\item Consider the circuit shown in Figure \ref{lab6_csource3}. Let $V_{CC} = \unit{10}{\volt}$ and $R = \unit{10}{\kilo\ohm}$. Roughly sketch $I_{OUT}$ vs. $V_{OUT}$.

\begin{figure}[!htb]
    \begin{center}
      \begin{asy}
	import graph;
	
	size(15cm,10cm,IgnoreAspect);

	scale(Linear,Linear);
	xlimits(0,20);
	ylimits(-1,1);
	
	pen thin=linewidth(0.5*linewidth());
	
	xaxis("$V_{OUT}$ (V)",BottomTop,LeftTicks(begin=false,end=false,extend=false,ptick=thin));
	yaxis("$I_{OUT}$ (mA)",LeftRight,RightTicks(begin=false,end=false));

	yequals(0,Dotted);

	xequals(5,Dotted);
	xequals(10,Dotted);
	xequals(15,Dotted);	
\end{asy}
    \end{center}
  \end{figure}

	\item Find the short circuit output current and the output impedance of the current source.
	\\~\\~\\~\\~\\~\\~\\~\\
	\item Is it possible to increase the output impedance without decreasing the output current and without changing $V_{CC}$? Explain.
	\\~\\~\\~\\~\\~\\~\\~\\
\end{enumerate}

\end{document}
