\documentclass{article}
\usepackage{ee105}

%% Makes figure labels bold
\usepackage[labelfont=bf]{caption}

\begin{document}

%% Prevent headings on the first page, since we're not using \maketitle
\thispagestyle{plain}

%% Standard header. Title of the lab goes here.
\report{6}{Biasing Circuitry}
\name

\section{Lab Questions}

\begin{enumerate}
	
	\item[3.1.3] What is $R_C$ when $V_{OUT}=\unit{650}{\milli\volt}$? \\ ~ \\
	$\boxed{R_{C}	= ~~~~~~~~~~~~~~~~~~~~~~~}$
	
	\item[3.1.4] Roughly sketch $I_{C}$ vs. $V_{OUT}$ for the transistor and for the resistor, showing the fixed point solution for $V_{OUT}$. How would we adjust the resistor to increase $V_{OUT}$?
	\\~\\~\\~\\~\\~\\~\\~\\
	
	\item[3.1.5] Will the voltage source become better or worse (better as defined by being closer to an ideal source) as the resistor decreases? Why?
	\\~\\~\\~\\~\\~\\~\\~\\
	
	\item[3.1.6] Find the output impedance of the voltage source.\\~\\
	$\boxed{R_{out}  	= ~~~~~~~~~~~~~~~~~~~~~~~}$
	
	\item[3.1.7] Now, suppose you want to make your voltage source output \unit{1.3}{\volt}. Clearly, putting \unit{1.3}{\volt} on $V_{BE}$ of the diode connected BJT is not a good idea (please, don't even try). Draw a circuit topology to achieve this voltage without requiring a BJT to have an extremely high $V_{BE}$.
	\\~\\~\\~\\~\\~\\~\\~\\

	\item[3.2.2] Short circuit current: \\ ~ \\
		$\boxed{I_{OUT}   = ~~~~~~~~~~~~~~~~~~~~~~~}$
	
	\item[3.2.3] Find $R_{out}$ in terms of the small-signal characteristics. \\~\\

		$\boxed{\text{Theoretical}~R_{out}   = ~~~~~~~~~~~~~~~~~~~~~~~}$
	
	\item[3.2.4] What happens to the output impedance as $V_{OUT}$ nears \unit{5}{\volt}?
		\\~\\~\\~\\~\\~\\~\\~\\
	
	\item[3.2.5] Output impedance at $V_{OUT}$ = \unit{2.5}{\volt} \\ ~ \\
		$\boxed{\text{Measured}~R_{out}   = ~~~~~~~~~~~~~~~~~~~~~~~}$
	
	\item[3.2.6--8] Transistors in parallel with $V_{OUT}$ = \unit{2.5}{\volt}: \\ ~ \\
		$\boxed{I_{OUT}   = ~~~~~~~~~~~~~~~~~~~~~~~}$ \\ ~ \\
		$\boxed{R_{out}   = ~~~~~~~~~~~~~~~~~~~~~~~}$ \\ ~ \\
		Explain the effect of the second transistor on the output impedance.
		\\~\\~\\~\\~\\~\\~\\~\\

	\item[3.3.2--6] Properties of the CE amp with current mirror: \\ ~ \\
		$\boxed{V_{IN}   = ~~~~~~~~~~~~~~~~~~~~~~~}$ \\ ~ \\
		$\boxed{A_{v}    = ~~~~~~~~~~~~~~~~~~~~~~~}$ \\ ~ \\
		$\boxed{I_{C2}	 = ~~~~~~~~~~~~~~~~~~~~~~~}$ \\ ~ \\
		$\boxed{I_{C3}	 = ~~~~~~~~~~~~~~~~~~~~~~~}$ \\ ~ \\		
    	$\boxed{R_{in}   = ~~~~~~~~~~~~~~~~~~~~~~~}$ \\ ~ \\
    	$\boxed{R_{out}  = ~~~~~~~~~~~~~~~~~~~~~~~}$

	\item[3.3.7] How do the impedances and gain compare with a common emitter biased with a resistor instead?
		\\~\\~\\~\\
		
	\item[3.3.8] Explain this effect using what you know about BJT temperature effects. How may this be an advantage of BJT biasing over resistive biasing?
		\\~\\~\\~\\~\\~\\~\\~\\

\end{enumerate}
\end{document}
