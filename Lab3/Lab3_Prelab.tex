\documentclass{article}
\usepackage{ee105}

%% Makes figure labels bold
\usepackage[labelfont=bf]{caption}

\begin{document}

%% Prevent headings on the first page, since we're not using \maketitle
\thispagestyle{plain}

%% Standard header. Title of the lab goes here.
\prelab{3}{Bipolar Junction Transistor Characterization}
\name

\begin{enumerate}
\item
  For the NPN device shown below in Figure \ref{prelab1}, fill in $I_C$, $I_B$, and $I_E$ next to the current arrows.
  
  \begin{figure}[!htb]
    \input lab3_prelab_bjt0
    \centerline{\box\graph}
    \caption{A simple NPN device for warming up}
    \label{prelab1}
  \end{figure}
    
\item What is $\beta$ in terms of $I_C$ and $I_B$? What is $\alpha$ in terms of $I_C$ and $I_E$? Express $\alpha$ in terms of $\beta$.
  \begin{align*}
    \boxed{\beta\left(I_C, I_B\right) = \frac{}{}~~~~~~~~~~~~~~~~~~~~~~~~~~~~~~~~~~~~~~~~~~~~} \\
    \boxed{\alpha\left(I_C, I_E\right) = \frac{}{}~~~~~~~~~~~~~~~~~~~~~~~~~~~~~~~~~~~~~~~~~~~~} \\
    \boxed{\alpha\left(\beta\right) = \frac{}{}~~~~~~~~~~~~~~~~~~~~~~~~~~~~~~~~~~~~~~~~~~~~} \\
  \end{align*}
\item
  SPICE

  \begin{figure}[!htb]
    \input lab3_prelab_bjt
    \centerline{\box\graph}
    \caption{Circuit to simulate in SPICE}
    \label{prelab2}
  \end{figure}

  \begin{itemize}
  \item Write a SPICE netlist for the BJT test circuit shown in Figure \ref{prelab2}. \textit{Refer to the \href{\baseurl/tutorials/HSPICE_Tutorial.pdf}{HSPICE Tutorial} if you have trouble with SPICE.}    
  \item Use the 2N4401 SPICE model provided on the course website.
  \item Using the \verb|.dc| command, sweep $V_{CC}$ from \unit{0}{\volt} to \unit{5}{\volt} in \unit{0.01}{\volt} increments and step $V_{BB}$ from \unit{0.6}{\volt} to \unit{0.7}{\volt} in \unit{0.025}{\volt} increments.
  \item Run the simulation and check the output file for any errors.  
  \item If there are no errors, plot $I_{CC}$ versus $V_{CC}$ and print out a copy of the plot. \note{If you notice that $I_{CC}$ is negative, use Awaves to plot the absolute value of $I_{CC}$. $I_{CC}$ appears to be negative because SPICE defines $I_{CC}$ to be going out of the BJT.}
  \end{itemize}
  
  
\item The configuration shown below in Figure \ref{darlington1} is known as the Darlington pair. Assume $Q_1$ has a DC current gain of $\beta_1$ and $Q_2$ has a DC current gain of $\beta_2$. Derive the overall current gain, $\beta_{tot} = I_{C2}/I_{B1}$, as a function of $\beta_1$ and $\beta_2$. Do not neglect any currents.
  
  \begin{figure}[!htb]
    \input lab3_prelab_darlington
    \centerline{\box\graph}
    \caption{Darlington configuration}
    \label{darlington1}
  \end{figure}

  ~\\~\\~\\~\\~\\~\\~\\~\\

  \begin{align*}
    \boxed{\beta_{tot} = ~~~~~~~~~~~~~~~~~~~~~~~~~~~~~~~~~~~~~~~~~~~~}
  \end{align*}
  
\end{enumerate}

\end{document}
