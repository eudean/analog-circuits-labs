\documentclass{article}
\usepackage{ee105}

%% Makes figure labels bold
\usepackage[labelfont=bf]{caption}

\begin{document}

%% Prevent headings on the first page, since we're not using \maketitle
\thispagestyle{plain}

%% Standard header. Title of the lab goes here.
\report{9}{MOS Characterization and Amplifiers}
\name

\begin{enumerate}
\item[3.1.2] Attach your printout.
\item[3.1.3] Approximately what criterion determines the boundary between saturation and triode?
  ~\\~\\~\\~\\~\\~\\~\\
\item[3.1.4] Properties (Part 1) \\ ~ \\
  $\boxed{g_m	= ~~~~~~~~~~~~~~~~~~~~~~~}$ \\ ~ \\
  $\boxed{r_o	= ~~~~~~~~~~~~~~~~~~~~~~~}$ \\ ~ \\
  $\boxed{\text{Region of Operation:} ~~~~~~~~~~~~~~~~~~~~}$
\item[3.1.5] Properties (Part 2) \\ ~ \\
  $\boxed{g_m	= ~~~~~~~~~~~~~~~~~~~~~~~}$ \\ ~ \\
  $\boxed{r_o	= ~~~~~~~~~~~~~~~~~~~~~~~}$ \\ ~ \\
  $\boxed{\text{Region of Operation:} ~~~~~~~~~~~~~~~~~~~~}$
\item[3.2.1] Channel Length Modulation Factor \\ ~ \\
	$\boxed{\lambda	= ~~~~~~~~~~~~~~~~~~~~~~~}$
\item[3.2.3] Attach plot of $(I_{D})^{\frac{1}{2}}$ vs. $V_G$.
\item[3.2.4] Find $K_n$. \\ ~ \\
  $\boxed{K_n	= ~~~~~~~~~~~~~~~~~~~~~~~}$
\item[3.2.5] Find $V_{TH}$. \\ ~ \\
  $\boxed{V_{T}	= ~~~~~~~~~~~~~~~~~~~~~~~}$
\item[3.3.2] Identify the two amplifier stages. \\ ~ \\
 \\~\\~\\~\\~\\~\\~\\
\item[3.2.3] Find the DC bias of $V_{IN}$ for maximum gain. Find the gain and output swing at this bias point. What problems might we encounter with biasing exactly at the point of maximum gain? \\ ~ \\
  $\boxed{V_{IN} = ~~~~~~~~~~~~~~~~~~~~~~~}$ \\ ~ \\
  $\boxed{A_v =    ~~~~~~~~~~~~~~~~~~~~~~~}$ \\ ~ \\
  $\boxed{\text{Output Voltage Swing: }~~~~~~~~~}$ \\ ~ \\ ~ \\ ~ \\ ~ \\
\item[3.3.4] What problems might we run into if the resistor were too big or too small? \\ ~ \\
  \\~\\~\\~\\~\\~\\~\\
\end{enumerate}

\end{document}
