\documentclass{article}
\usepackage{ee105}

%% Makes figure labels bold
\usepackage[labelfont=bf]{caption}

\begin{document}

%% Prevent headings on the first page, since we're not using \maketitle
\thispagestyle{plain}

%% Standard header. Title of the lab goes here.
\prelab{4}{Single Stage BJT Amplifiers: Common Emitter}
\name

\begin{figure}[!htb]
	\input lab4_bjt
	\centerline{\box\graph}
	\caption{Common emitter amplifier}
	\label{CEA_prelab}
\end{figure}

\begin{enumerate}

	\item Let's analyze this common emitter amplifier! See Figure \ref{CEA_prelab}. Let $V_{CC} = \unit{5}{\volt}$, $V_{T} = \unit{26}{\milli\volt}$, $I_{S} = \unit{26.03}{\femto\ampere}$, $V_{A} = \unit{90.7}{\volt}$, $R_{C} = \unit{1}{\kilo\ohm}$, and $\beta = 270$.

	\begin{enumerate}

		\item For this transistor draw a graph (found on the next page) of $I_{C}$ vs. $V_{CE}$ for $V_{BE}$ = \unit{600}{\milli\volt}, \unit{620}{\milli\volt}, \unit{640}{\milli\volt}, \unit{660}{\milli\volt}, \unit{680}{\milli\volt}, and \unit{700}{\milli\volt}. Take care drawing this graph as it will be used in answering some of the following questions. Assume the boundary between deep saturation and forward active occurs at $V_{CE} = \unit{400}{\milli\volt}$. You can use a piecewise linear model for this graph.
		
		%% here we will insert some asymptote graphs once I learn how to draw them

		\begin{figure}[!htb]
		  \begin{center}
		    \begin{asy}
		      import graph;
		      texpreamble("\def\Arg{\mathop {\rm Arg}\nolimits}");
		      
		      size(15cm,8cm,IgnoreAspect);
		      
		      scale(Linear,Linear);
		      ylimits(0,15);
		      xlimits(0,5);
		      
		      pen xthin=linewidth(0.5*linewidth());
		      pen ythin=linewidth(0.5*linewidth());
		      
		      xaxis("$V_{CE}$ (V)",BottomTop,LeftTicks(begin=false,end=false,extend=false,ptick=xthin));
		      yaxis("$I_{C}$ (mA)",LeftRight,RightTicks(begin=false,end=false,extend=false,ptick=ythin,N=0,n=0,Step=3,step=1));
		      
		      yequals(3,Dotted);
		      yequals(6,Dotted);
		      yequals(9,Dotted);
		      yequals(12,Dotted);
		      
		      xequals(1,Dotted);
		      xequals(2,Dotted);
		      xequals(3,Dotted);
		      xequals(4,Dotted);
		      xequals(5,Dotted);
\end{asy}
		  \end{center}
		\end{figure}

		\item On this graph, draw the load line for $R_{C}$ in this circuit.

		%% same

		\item Using the intersection points, draw a graph (also found on the next page) of $V_{OUT}$ vs. $V_{IN}$. What is this graph useful for?
		~\\~\\

		%% same
		\begin{figure}[!htb]
		  \begin{center}
		    \begin{asy}
		      import graph;
		      texpreamble("\def\Arg{\mathop {\rm Arg}\nolimits}");
		      
		      size(15cm,10cm,IgnoreAspect);
		      
		      scale(Linear,Linear);
		      ylimits(0,5);
		      xlimits(600,700);
		      
		      pen xthin=linewidth(0.5*linewidth());
		      pen ythin=linewidth(0.5*linewidth());
		      
		      xaxis("$V_{IN}$ (mV)",BottomTop,LeftTicks(begin=false,end=false,extend=false,ptick=xthin,N=0,n=0,Step=10,step=5));
		      yaxis("$V_{OUT}$ (V)",LeftRight,RightTicks(begin=false,end=false,extend=false,ptick=ythin));
		      
		      yequals(1,Dotted);
		      yequals(2,Dotted);
		      yequals(3,Dotted);
		      yequals(4,Dotted);
		      
		      xequals(620,Dotted);
		      xequals(640,Dotted);
		      xequals(660,Dotted);
		      xequals(680,Dotted);
		      xequals(700,Dotted);
\end{asy}
		  \end{center}
		\end{figure}

		\item What is the gain for $V_{IN} = \unit{650}{\milli\volt}$? What is the region of operation of the transistor?
		~\\~\\

	\end{enumerate}

	\item Now that we've analyzed the large-signal properties of the amplifier, let's analyze some of its small-signal properties. Assume $V_{IN} = \unit{650}{\milli\volt}$ for the rest of the prelab.

	\begin{enumerate}

		\item Draw the small signal model of the amplifier (answer space provided on next page).
		~\\~\\~\\~\\~\\~\\~\\~\\~\\

		\item Calculate the following (be sure to include the Early effect):
			\begin{itemize}
			\item $V_{OUT}$
			\item $I_C$
			\item Transconductance $g_{m}$
			\item Output impedance $R_{out}$
			\item Input impedance $R_{in}$
			\item Gain $A_{v}$. Does this gain match your answer to question 1(d)?
			\end{itemize}
			~\\~\\~\\~\\~\\~\\~\\~\\~\\~\\~\\~\\~\\~\\~\\

		\item If we change the bias point, which of the above properties change?
		~\\~\\~\\~\\~\\~\\
	\end{enumerate}

	\item Let's explore what happens when we change the bias point by adding a load resistor.
	
	\begin{enumerate}
		
		\item Add a load of \unit{5}{\kilo\ohm} to the output of the circuit. Intuitively, how do you think this will affect the gain and why?
		~\\~\\~\\~\\

		\item Draw the new load line on the graph you made for question 1(a) (be sure to label which load line goes with which question).

		\item If we keep the same bias point (i.e., the same $V_{BE}$ value), is the transistor still in forward active mode?
		~\\~\\~\\

		\item What happens to the gain? How about the output voltage swing? \note{The output voltage swing is the maximum and minimum values of $v_{OUT}$ for which all transistors stay in forward active.}
		~\\~\\~\\

		\item What is the load resistance value for which the amplifier's BJT begins to transition between forward active mode and deep saturation (i.e., where $V_{CE} = \unit{400}{\milli\volt}$)?
		~\\~\\~\\~\\~\\

	\end{enumerate}
	
\end{enumerate}

\end{document}
