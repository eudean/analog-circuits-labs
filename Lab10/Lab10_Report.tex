\documentclass{article}
\usepackage{ee105}

%% Makes figure labels bold
\usepackage[labelfont=bf]{caption}

\begin{document}

%% Prevent headings on the first page, since we're not using \maketitle
\thispagestyle{plain}

%% Standard header. Title of the lab goes here.
\report{10}{Differential Amplifiers}
\name

\begin{enumerate}
  \item[3.2.2] Measure $I_{C1}$, $I_{C2}$, $I_{C3}$, and $V_{OUT,DC}$. How do they compare with hand calculations?
    \begin{align*}
      \boxed{I_{C1} = ~~~~~~~~~~~~~~~~~~~~~~ } \\
      \boxed{I_{C2} = ~~~~~~~~~~~~~~~~~~~~~~ } \\
      \boxed{I_{C3} = ~~~~~~~~~~~~~~~~~~~~~~ } \\
      \boxed{V_{OUT,DC} = ~~~~~~~~~~~~~~~~~~~~~~ } \\
    \end{align*}
    \\
  \item[3.2.3] Sketch the waveforms at $v_{in+}$ and $v_{out+}$.

\begin{figure}[!htb]
\begin{center}
\begin{asy}
import math;
size(250,0);

add(shift(-5,-4)*grid(10,8));

dot((0,0),black);
\end{asy}
\end{center}
\end{figure}
  \item[3.2.4] Measure the peak-to-peak voltages of $v_{in+}$ and $v_{out+}$
    \begin{align*}
      \boxed{v_{in+,p-p} = ~~~~~~~~~~~~~~~~~~~~~~ } \\
      \boxed{v_{out+,p-p} = ~~~~~~~~~~~~~~~~~~~~~~ } \\
    \end{align*}
  \item[3.2.5] Qualitatively describe how $v_{out+}$ and $v_{out-}$ are related. Is this what you'd expect?
    \\~\\~\\~\\~\\
  \item[3.2.6] Measure the peak-to-peak voltage of $v_{out+} - v_{out-}$ and calculate the differential gain of the circuit. Does this match the gain you calculated in the prelab?
    \begin{align*}
      \boxed{v_{out,p-p} = ~~~~~~~~~~~~~~~~~~~~~~ } \\
      \boxed{A_{DM} = ~~~~~~~~~~~~~~~~~~~~~~ } \\
    \end{align*}
    \\
  \item[3.2.7] What do you see at the output? Why?
    \\~\\~\\~\\~\\
  \item[3.2.8] Measure the gain. Does it match your prelab calculations? Does it match your result from 3.2.6?
    \begin{align*}
      \boxed{A_{DM} = ~~~~~~~~~~~~~~~~~~~~~~ } \\
    \end{align*}
    \\
  \item[3.3.2] Sketch the output waveform. Why isn't it sinusoidal?
\begin{figure}[!htb]
\begin{center}
\begin{asy}
import math;
size(250,0);

add(shift(-5,-4)*grid(10,8));

dot((0,0),black);
\end{asy}
\end{center}
\end{figure}
\\
  \item[3.3.4] Calculate the differential gain of the amplifier with the added load.
    \\~\\~\\~\\
    \begin{align*}
      \boxed{A_{DM} = ~~~~~~~~~~~~~~~~~~~~~~ } \\
    \end{align*}
  \item[3.3.5] Sketch $v_{out}$. What is the measured differential gain of the circuit? How does it compare to your hand calculations? Does it match the gain you observed in step 3.2.6? Should it?
\begin{figure}[!htb]
\begin{center}
\begin{asy}
import math;
size(250,0);

add(shift(-5,-4)*grid(10,8));

dot((0,0),black);
\end{asy}
\end{center}
\end{figure}
    \begin{align*}
      \boxed{A_{DM} = ~~~~~~~~~~~~~~~~~~~~~~ } \\
    \end{align*}
    \\~\\~\\~\\~\\

  \item[3.4.1] Attach your netlist on a separate sheet.
  \item[3.4.2] Use SPICE to find $I_{C1}$, $I_{C2}$, $I_{C3}$, and $V_{out,DC}$. Compare these values with your calculations from the prelab and measurements in lab.
    \begin{align*}
      \boxed{I_{C1} = ~~~~~~~~~~~~~~~~~~~~~~ } \\
      \boxed{I_{C2} = ~~~~~~~~~~~~~~~~~~~~~~ } \\
      \boxed{I_{C3} = ~~~~~~~~~~~~~~~~~~~~~~ } \\
      \boxed{V_{OUT,DC} = ~~~~~~~~~~~~~~~~~~~~~~ } \\
    \end{align*}
    \\~\\~\\~\\~\\
  \item[3.4.3] Attach your plot on a separate sheet. What is the gain as measured from the plot? Does it match your hand calculations? Does it match your measurements?
    \begin{align*}
      \boxed{A_{DM} = ~~~~~~~~~~~~~~~~~~~~~~ }
    \end{align*}
    \\~\\~\\
\end{enumerate}

\end{document}